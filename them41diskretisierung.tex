\section{Diskretisierung und Aufstellung des Gleichungssystems}
Da das gegebene Problem im Allgemeinen nicht analytisch lösbar ist, soll es im Folgenden numerisch gelöst werden. Hierfür wird der untersuchte Bereich $D$ durch konstante Schrittweiten $h$ diskretisiert, was uns zu der neuen Fläche führt, die von den Variablen $a=n\,h$ $b=m\, h$ aufgespannt wird. und nun als $D_h=\{(x,y)\in D | x=nh, y=mh ; n,m \in \mathbb{Z}\} \partial D_h=\{(x,y) \in \partial D | x \in \{ nh, 0\} y= mh, m\in \mathbb{Z}; y \in \{0, mh\} x=nh, n\in \mathbb{Z}\}$ geschrieben wird.
An diese Diskretisierung wird nun die Differentialgleichung angepasst, indem die Ableitungen als mit der Methode der zentralen Differenzen aproximiert werden. Damit wird diese zu 
\[\frac{\partial u(x,y)}{\partial x} \approx \frac{u(x+h, y)-u(x-h,y)}{2h}\]
Analog sieht die Ableitung nach y aus. Hiermit folgt nun:\\
\[\frac{\partial^2 u(x,y)}{\partial x^2}=\frac{u(x+h,y)+u(x-h,y)-2u(x,y)}{2h}\]
\[\frac{\partial^2 u(x,y)}{\partial y^2}=\frac{u(x,y+h)+u(x,y-h)-2u(x,y)}{2h}\]
Nun folgt die obige DGL: (in den Funktionen jew. mit Schrittweite h multiplizieren!!!)
$\frac{u(i+1,j)+u(i-1,j)-2u(i,j)}{h^2}+\frac{u(i,j+1)+u(i,j-1)-2u(i,j)}{h^2}=-k f(x,y)$ für $i,j \in D_h=\{(ih,jh)|i=1,...,n-1; j=1,...,m-1$
