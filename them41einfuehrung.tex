\section{Einf"uhrung}
Jede reale Platte ist, wenn auch meist nicht im sichtbaren Bereich, verformbar und gibt dem entsprechend bei Belastung durch ein Gewicht wie eine Vase, ein Stuhl oder "ahnlichem, nach. Wie diese Verformung dabei aussieht, soll im Folgenden gekl"art werden. Dabei wird eine rechteckige Platte betrachtet, die am Rand so verst"arkt ist, dass sie sich hier nicht verformen kann. Daraus ergibt sich das Randwertproblem
\begin{align}\frac{\partial ^2}{\partial x^2}u(x,y)+\frac{\partial ^2}{\partial y^2}u(x,y)=-k\,f(x,y) \quad x,y \in D \\
u(x,y)=0 \forall x,y \in \partial D \end{align}
, die so genannte \glqq Poisson-Gleichung \grqq. Hier beschreibt $u$ die Verformung der Platte, sodass $\Delta u$ die Kr"ummung %eigentlich nicht!?
beschreibt. Diese wird von der Gewichtsfunktion $f$ erzeugt. $k$ ist eine Skalierungsfunktion, welche materialspezifische Gr"oßen beinhaltet. Sie wird im Folgenden meist einfach auf $k=1$ gesetzt, da sie f"ur die grunds"atzliche Betrachtung des Problems keine Rolle spielt.\\
Wie bereits aus den Gleichungen (1) deutlich wird, erstreckt sich die Platte "uber ein Gebiet $D$, welches, ohne Beschr"ankung der Allgemeinheit als $D=(0,a)\times (0,b)$  definiert wird.
\subsection{L"osbarkeit des Problems}

